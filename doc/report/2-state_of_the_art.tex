\chapter{State of the art}

An overall of four different well-known tasks can be detected in the problem, namely:

\begin{itemize}

  \item \textbf{Arena exploration}: the robots must explore the arena to find the landmarks.
  \item \textbf{Clustering}: the robots must create cluster near any landmark so it can be considered explored.
  \item \textbf{Phototaxis}: used to allow the robots to easily come back to the base.
  \item \textbf{Obstacle avoidance}: general behaviour to be executed to prevent any collision (between robots as well as between robots and obstacles/walls).

\end{itemize}

\noindent
It follows a brief description of the state of the art for each task.

\section{Arena exploration}

The exploration of the area is a well-known task in the literature, and lots of algorithms and methodologies have been studied. However, when talking about swarm robotics, the main adopted method is the \textit{random walk} approach. This is mainly due to the usage of off-the-shelf robots that own limited individual abilities (low processing power, local sensing etc.). Nevertheless, the usage of a swarm of robots for exploration tasks is still widely applied thanks to the advantages it entails respect to the usage of a single robot (flexibility, robustness, scalability).

\bigskip
There are few different common sense implementations of a random walk task \cite{rw-summary}, namely:

\begin{itemize}

  \item {Brownian motion}
  
  \item {Lèvy Flight}
  
  \item {Lèvy Taxis}

  \item {Correlated radom walk}

  \item{Ballistic motion}  
\end{itemize}
 
The main idea is to randomly choose a direction and go straight till a new one is selected.

\noindent
All the aforementioned models share the same underlying mathematical model, and the different tuning of the parameters provides different behaviours and exploration capabilities. Only one exception can be identified, that is the ballistic motion, where the two parameters \textit{step length} ($\mu$) and \textit{turning angle} ($\rho$) are not provided and/or limited. 

\bigskip
Further from the above basic approaches, new solution have been probed to overcome the limits they impose. Indeed, there are two main problem that emerge, that is:

\begin{itemize}

  \item Execution of repeated exploration of the same point.
  \item Exploration does not scale with arena widening. Ineed, just zona near the starting position will be quite well explored.
 
\end{itemize}

New approaches have been studied to overcome the aboe issues. The example reported in \cite{rw-improved} improves the random walk by considering relative distribution of robots among the arena so that each zone can be equally explored by a even quantity of robots.  

\bigskip
Once the robots are spread over the arena, different algorithm can be applied to map it (e.g., \textit{GMapping} that produces a two-dimensional occupancy grid of the environment). 

\section{Clustering}

Clustering is one of the main tasks identified in most of the proposed taxonomies for swarm robotics. The goal is to gather objects (the robot themselves or tokens to be moved) in one or more points of the arena. The task can be used for multiple purposes: foraging activity (group food supplies), divide robots upon classes that execute different tasks, etc..

\noindent
All the proposed clustering algorithms create different solutions upon one of three common approaches \cite{clustering-summary}, that is:

\begin{itemize}

  \item Force-oriented solutions (similar to motor schema principles).
  \item Probabilistic approach by using non-deterministic transition among states and tasks. Indeed, as stated in \cite{clustering-randomness} \cite{clustering-data}, randomness resulted to be a crucial aspect for emerging behaviours in swarm robotics in general. 
  \item Artificial evolution through genetic algorithms. 

\end{itemize}

Independently from the ad hoc approach, the main idea is to apply \textit{swarm intelligence} principles to guide robots behaviour, and so to create a result inspired to various animal species (insects, birds, fishes) or natural laws (e.g., the settling process of liquids of different densities, \cite{clustering-natural}). 

\noindent
Beyond this, few more articulated experiments have been executed trying to add further capabilities to robots (e.g., \textit{spatial awareness} and exchange of \textit{virtual tokens}, \cite{clustering-awareness}).

\noindent
In the end, notice that swarm intelligence principles can be applied to other domains as well. Indeed, lots of experiments have been made about the clustering of data (instead of robots) in order to overcome traditional clustering algorithms such as \textit{K-means} by blending swarm intelligence with other domains (e.g., game theory, \cite{clustering-data}). 

\section{Obstacle avoidance}

The obstacle avoidance task has been widely studied, and so few different algorithms have been proposed. 
The input values can be retrieved with distance sensors (e.g., sonar, proximity sensors) and/or visual sensors. In this case, we are going to focus on techniques that exploit the first ones solely.

\smallskip
The most known technique is the \textit{Artificial Potential Field (APF)} that is based upon the motor schema idea by creating two potential fields: the first is attractive and generated by the goal point so that the robot can aim toward it, the second one is repulsive and it is generated from obstacles. Obviously, even though this method suffers all the issues of the motor schema approach such as local minima, it finds shorter paths than other well-know algorithm such as the \text{Bug} one \cite{obstacle-summary} \cite{obstacle-summary-2}.

\noindent
Other solutions, even though rely on the same underlying mathematical model, contemplate different types of potential fields: tangential fields to circumnavigate obstacles, Gaussian potential fields \cite{obstacle-gaussian}.

\noindent
In addition, we notice the \textit{Vector Field Histogram} algorithm that produces a vector force as output (length, angle), but differently from others it exploits peculiar tools as histograms to represent the distribution of obstacles around the robot. The algorithm provides for three stages so that the initial 2D histogram is collapsed to a polar one.  

\noindent
Moreover, as well as the classical approaches, novel solutions have been explored by combining genetic algorithms and neural networks (as reported in \cite{obstacle-NN}).

\noindent
In the end, complete navigation with obstacle avoidance have been experimented by using the \textit{Particle Swarm Optimization} algorithm \cite{obstacle-summary-2}.




